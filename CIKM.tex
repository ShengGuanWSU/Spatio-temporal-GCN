\documentclass[sigconf]{acmart}

\usepackage{graphicx}
\usepackage{balance}  % for  \balance command ON LAST PAGE  (only there!)
% added according to previous template
%\usepackage{cite}
%cite package conflict with natbib package
\usepackage{amsmath,amssymb,amsfonts}
\usepackage{algorithmic}
\usepackage{graphicx}
\usepackage{textcomp}
%\usepackage[svgnames]{xcolor}
\usepackage{latexsym}
\usepackage{amsfonts}
\usepackage{amsmath}
\usepackage{amssymb}
\usepackage{color}
\usepackage{epsfig}
\usepackage{xspace}
\usepackage{graphicx}
\usepackage{subfigure}
\usepackage{balance}
\usepackage{rotating}
\usepackage{pbox}
\usepackage{caption}
\usepackage{epsfig}
\usepackage{multirow}
\usepackage{url}
\usepackage{mathrsfs}
%\usepackage[implicit=false]{hyperref}
%\usepackage{adjustbox}
\usepackage{booktabs}
\usepackage{eufrak}

\input{commands}

%%
%% \BibTeX command to typeset BibTeX logo in the docs
\AtBeginDocument{%
  \providecommand\BibTeX{{%
    \normalfont B\kern-0.5em{\scshape i\kern-0.25em b}\kern-0.8em\TeX}}}

%% Rights management information.  This information is sent to you
%% when you complete the rights form.  These commands have SAMPLE
%% values in them; it is your responsibility as an author to replace
%% the commands and values with those provided to you when you
%% complete the rights form.
\setcopyright{acmcopyright}
\copyrightyear{2018}
\acmYear{2018}
\acmDOI{10.1145/1122445.1122456}

%% These commands are for a PROCEEDINGS abstract or paper.
\acmConference[Woodstock '18]{Woodstock '18: ACM Symposium on Neural
  Gaze Detection}{June 03--05, 2018}{Woodstock, NY}
\acmBooktitle{Woodstock '18: ACM Symposium on Neural Gaze Detection,
  June 03--05, 2018, Woodstock, NY}
\acmPrice{15.00}
\acmISBN{978-1-4503-XXXX-X/18/06}

\settopmatter{printacmref=false}
\setcopyright{none}
\renewcommand\footnotetextcopyrightpermission[1]{}
\pagestyle{plain}

%%
%% Submission ID.
%% Use this when submitting an article to a sponsored event. You'll
%% receive a unique submission ID from the organizers
%% of the event, and this ID should be used as the parameter to this command.
%%\acmSubmissionID{123-A56-BU3}

%%
%% The majority of ACM publications use numbered citations and
%% references.  The command \citestyle{authoryear} switches to the
%% "author year" style.
%%
%% If you are preparing content for an event
%% sponsored by ACM SIGGRAPH, you must use the "author year" style of
%% citations and references.
%% Uncommenting
%% the next command will enable that style.
%%\citestyle{acmauthoryear}

%%
%% end of the preamble, start of the body of the document source.
\begin{document}


%%
%% The "title" command has an optional parameter,
%% allowing the author to define a "short title" to be used in page headers.
\title{Spatio-temporal GCN} %Incomplete

%%
%% The "author" command and its associated commands are used to define
%% the authors and their affiliations.
%% Of note is the shared affiliation of the first two authors, and the
%% "authornote" and "authornotemark" commands
%% used to denote shared contribution to the research.
%\author{Sheng Guan$^*$,\ \ \ Hanchao Ma$^*$, \ %\ \ Yinghui Wu$^*$ \ \ \ }
%\affiliationn{Case Western Reserve %University$^*$}
%\email{{sxg967,\hspace{0.2em}hxm382,\hspace{0.2%em}yxw1650}@case.edu$^*$}


\eat{
\author{Sheng Guan$^*$ \ \ \   Hanchao Ma$^*$ \ \ \    Yinghui Wu$^*$ \ \ \ }
\affiliation{Case Western Reserve University$^*$ }
\email{{sxg967,hxm382,yxw1650}@case.edu}
}

%% By default, the full list of authors will be used in the page
%% headers. Often, this list is too long, and will overlap
%% other information printed in the page headers. This command allows
%% the author to define a more concise list
%% of authors' names for this purpose.
%\renewcommand{\shortauthors}{Trovato and Tobin, et al.}

%%
%% The abstract is a short summary of the work to be presented in the
%% article.
\begin{abstract}
Abstract goes here. 
%%%%%%%%%%%%%%%%%%%%%%%%%%%%%
%%%%%%%%% for site submission
\end{abstract}

%%
%% The code below is generated by the tool at http://dl.acm.org/ccs.cfm.
%% Please copy and paste the code instead of the example below.
%%

%% A "teaser" image appears between the author and affiliation
%% information and the body of the document, and typically spans the
%% page.


%%
%% This command processes the author and affiliation and title
%% information and builds the first part of the formatted document.
\maketitle



\stitle{Related Work.} We categorize the related work as follows.
GeniePath: Graph Neural Networks with adaptive receptive paths \cite{liu2019geniepath}.
KeyIdea:
GeniePath, a scalable approach for learning adaptive receptive fields of neural networks defined on permutation invariant graph data. 
Adaptive path layer consists of two complementary functions: 
(1)	Breadth exploration—learns the importance of different sized neighborhoods
(2)	Depth exploration –extracts and filters signals aggregated from neighbors of different hops away.
Dataset:
Pubmed (citation network, edges undirected citations)
BlogCatalog  (social networks, nodes correspond to bloggers and edges to social relationships)
one-hot features for each node 10,312 dimensional features-> decomposed by SVD on adjacency matrix A to get 128 dimensional features
Alipay dataset:  nodes correspond to users’ accounts and devices during a time period
		Node features: counts of login behaviors discretized into hours and account profiles
2 classes of nodes (malicious accounts and normal accounts)
82,246 disjoint subgraphs
PPI dataset

Compared methods:
(1)	Multilayer perceptron: simple embedding, considers no structures of the graph
(2)	Node2Vec (cannot work on multiple graphs: Alipay and PPI)
(3)	Chebyshev, one graph spectral convolutions 
(4)	GCN 
(5)	GraphSAGE
(6)	GAT

Motivation for residual learning \cite{he2016deep}: 
The degradation (of training accuracy) indicates that not all systems are similarly easy to optimize. Let us consider a shallower architecture and its deeper counterpart that adds more layers onto it. There exists a solution by construction to the deeper model: the added layers are identity mapping, and the other layers are copied from the learned shallower model. The existence of this constructed solution indicate that a deeper model should produce no higher training error than its shallower counterpart. But experiments show that our current solvers on hand are unable to find solutions that are comparably good or better than the constructed solution (or unable to do so in feasible time).
Solution:
Instead of hoping each few stacked layers directly fit a desired underlying mapping, we explicitly let these layers fit a residual mapping. Formally, denoting the desired underlying mapping as H(x), we let the stacked nonlinear layers fit another mapping of ${F(x) := H(x)- x}$. The original mapping is recast into ${F(x)+x}$. We hypothesize that it is easier to optimize the residual mapping than to optimize the original mapping.

Topology Attack and Defense for Graph Neural Networks:An Optimization Perspective \cite{xu2019topology} proposed 

When comparing to current adversarial attacks on GNNs, the results show that by only perturbing a small number of edge perturbations, including addition and deletion, our optimization-based attack can lead to a noticeable decrease in classification performance.






\section{Preliminary}
\label{sec-pre}

%\vspace{-1ex}
%We start with the notions of graphs.
\subsection{Temporal Graph and Learning Tasks} 

\stitle{Weighted Temporal Graph}. A weighted temporal graph 
$\mathcal{G}_T$ is a  
tuple $(V,E,W,L,T,F_A)$, where (1) $T$ is a time window (a sequence of consecutive timestamps); (2) $V$ is a set of nodes;
(3) ${E \subseteq V \times V \times T}$ is a set of edges associated with a timestamp from $T$; (4) an adjacency matrix $W$ assigns a value to each edge ${e \in E}$. The lower value of weight indicates that the endpoints are closer to each other; (5) function $L$ assigns a label $L(v)$ (resp. $L(e)$) to each node $v \in V$ (resp. edge ${e \in E}$). An edge ${e = (u_1, u_2, t)(t \in T)}$ encodes a link with label $L(e)$ between $u_1$ and $u_2$ that exists at timestamp $t$; and (6) Each node $v\in V$ has a
{\em node tuple} $F_A(v)$ =
$\{(A_1, a_1), \ldots, (A_n,a_n)\}$
defined on a set of node attributes $\A$,
where a pair $(A_i, a_i)\in F_A(v)$
states that the attribute $v.A_i\in\A$ has
a value $a_i\in\adom(A_i)$. Here
(a) $\A$ refers to a set of all the node
attributes seen in $G$; and (b)
$\adom(A_i)$ is a finite {\em active domain} of
attribute $A_i$ in $G$, and contains
all the values of $v.A_i$, where $v$ ranges
over all the nodes in $V$.

Given a timestamp ${t \in T}$, a snapshot $G_t$ of $\mathcal{G}_T$ at $t$ is a graph induced by the set of all the edges associated with timestamp $t$.
\input{sec-model}
\input{sec-framework}
\input{sec-expt}
\input{sec-conclude}




%%
%% The acknowledgments section is defined using the "acks" environment
%% (and NOT an unnumbered section). This ensures the proper
%% identification of the section in the article metadata, and the
%% consistent spelling of the heading.
\eat{
\begin{acks}
To Robert, for the bagels and explaining CMYK and color spaces.
\end{acks}
}
%%
%% The next two lines define the bibliography style to be used, and
%% the bibliography file.
\bibliographystyle{ACM-Reference-Format}
\bibliography{paper}

%%
%% If your work has an appendix, this is the place to put it.


\end{document}
%%
%% End of file `sample-sigconf.tex'.
